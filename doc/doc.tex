%%%%%%%%%%%%%%%%%%%%%%%%%%%%%%%%%%%%%%%%%%%%%%%%%%%%%%%%%%%%%%%%%%%%%%%%%%%%%%
%%%%%%%%%%%%%%%%%%%%%%%%%%%%%%%%%%%%%%%%%%%%%%%%%%%%%%%%%%%%%%%%%%%%%%%%%%%%%%
%%
%% Ukázkový příklad dokumentace úkolu do předmětů IZP a IUS, 2010
%%
%% Upravená původní dokumentace od Davida Martinka.
%%%%%%%%%%%%%%%%%%%%%%%%%%%%%%%%%%%%%%%%%%%%%%%%%%%%%%%%%%%%%%%%%%%%%%%%%%%%%%
%%%%%%%%%%%%%%%%%%%%%%%%%%%%%%%%%%%%%%%%%%%%%%%%%%%%%%%%%%%%%%%%%%%%%%%%%%%%%%
\documentclass[12pt,a4paper,titlepage,final]{article}

% cestina a fonty
\usepackage[slovak]{babel}
\usepackage[utf8]{inputenc}
\usepackage{etoolbox}


% balicky pro odkazy
\usepackage[bookmarksopen,colorlinks,plainpages=false,urlcolor=blue,unicode]{hyperref}
\usepackage{url}
% obrazky
\usepackage[dvipdf]{graphicx}
\usepackage{amsmath}

% velikost stranky
\usepackage[top=3.5cm, left=2.5cm, text={17cm, 24cm}, ignorefoot]{geometry}

\begin{document}
	
\newcommand{\defauthor}[3]{
	\expandafter\newcommand\csname #1-name\endcsname{#2}
	\expandafter\newcommand\csname #1-email\endcsname{#3}
}

%%%%%%%%%%%%%%%%%%%%%%%%%%%%%%%%%%%%%%%%%%%%%%%%%%%%%%%%%%%%%%%%%%%%%%%%%%%%%%
% titulní strana

% default author
\def\defaultauthor{tibor}

% mena autorov
\defauthor{tibor}{Tibor Mikita}{xmikit01@stud.fit.vutbr.cz}
\defauthor{martin}{Martin Matejčík}{xmatej46@stud.fit.vutbr.cz}

\ifdefined\whos
\else
\def\whos{\defaultauthor}
\fi

\def\myauthor{\csname\whos-name\endcsname}
\def\myemail{\csname\whos-email\endcsname}

\def\projname{Program pre zobrazovanie\\vlastností obrazu}
% !!!!!!!!!!!!!!!!!!!!!!!!!!!!!!!!!!!!!!!!!!!!!!!!

\begin{titlepage}

% \vspace*{1cm}
\begin{figure}[!h]
	\centering
  \includegraphics[height=3cm]{img/logo.pdf}
\end{figure}

\vfill

\begin{center}
\begin{Large}
Technická správa\\
\end{Large}
\bigskip
\begin{Huge}
\projname\\
\end{Huge}

\end{center}

\vfill

\begin{center}
\begin{Large}
\today
\end{Large}
\end{center}

\vfill

\begin{flushleft}
\begin{large}
\begin{tabular}{ll}
Autor: & \myauthor, \url{\myemail} \\
%& \mycoauthor, \url{\mycoemail} \\
 & Fakulta Informačních Technologií \\
 & Vysoké Učení Technické v~Brně \\
\end{tabular}
\end{large}
\end{flushleft}
\end{titlepage}


%%%%%%%%%%%%%%%%%%%%%%%%%%%%%%%%%%%%%%%%%%%%%%%%%%%%%%%%%%%%%%%%%%%%%%%%%%%%%%
% obsah
\pagestyle{plain}
\pagenumbering{roman}
\setcounter{page}{1}
\tableofcontents

%%%%%%%%%%%%%%%%%%%%%%%%%%%%%%%%%%%%%%%%%%%%%%%%%%%%%%%%%%%%%%%%%%%%%%%%%%%%%%
% textova zprava
\newpage
\pagestyle{plain}
\pagenumbering{arabic}
\setcounter{page}{1}

%%%%%%%%%%%%%%%%%%%%%%%%%%%%%%%%%%%%%%%%%%%%%%%%%%%%%%%%%%%%%%%%%%%%%%%%%%%%%%
\section{Zloženie tímu} \label{Autori}
Na tomto projekte pracovali, respektíve pracujú, respektíve ešte pracovať budú Tibor Mikita a Martin Matejčík.


%%%%%%%%%%%%%%%%%%%%%%%%%%%%%%%%%%%%%%%%%%%%%%%%%%%%%%%%%%%%%%%%%%%%%%%%%%%%%%
\section{Zadanie}
Zadaním je vypracovať program, ktorý bude schopný zobrazovať rôzne údaje o načítanom obrázku.

%%%%%%%%%%%%%%%%%%%%%%%%%%%%%%%%%%%%%%%%%%%%%%%%%%%%%%%%%%%%%%%%%%%%%%%%%%%%%%
\section{Náš cieľ}\label{Ciel}
Výsledkom tohto projektu bude aplikácia s grafickým uživateľským rozhraním, do ktorej bude možné nahrať ľubovoľný obrázok. Po načítaní obrázka do aplikácie sa, v uživateľsky prívetivom prostredí, zobrazia rôzne informácie. Budú to informácie týkajúce sa súboru ako takého, ale aj informácie súvisiace so samotným obrázkom, prípadne ďalšie informácie, ako je napríklad informácia o pixeli v mieste kurzora myši, histogram obrázka a iné.

%%%%%%%%%%%%%%%%%%%%%%%%%%%%%%%%%%%%%%%%%%%%%%%%%%%%%%%%%%%%%%%%%%%%%%%%%%%%%%
\section{Štúdium problematiky}
\ifdefined\whos
\input{\whos.tex}
\fi


%%%%%%%%%%%%%%%%%%%%%%%%%%%%%%%%%%%%%%%%%%%%%%%%%%%%%%%%%%%%%%%%%%%%%%%%%%%%%%
\section{Rozdelenie práce v tíme}
Martin:
\begin{itemize}
	\item základná štruktúra aplikácie podľa MVC modelu
	\item prepojenie použitých technológií
	\item rozdelenie obrázku na jednotlivé kanály a ich vzájomné zobrazovanie
\end{itemize}
Tibor:
\begin{itemize}
	\item načítanie obrázku zo súboru a získanie základných informácií o obrázku a o súbore ako takom
	\item histogram
	\item zobrazovanie informácií o pixele pod kurzom myši
	\item nahromadenie testovacej sady obrázkov
\end{itemize}

%%%%%%%%%%%%%%%%%%%%%%%%%%%%%%%%%%%%%%%%%%%%%%%%%%%%%%%%%%%%%%%%%%%%%%%%%%%%%%
\section{Použité technológie}
\begin{itemize}
	\item Java 8
	\item JavaFX (GUI)
	\item OpenCV 3.2 (práca s obrázkami)
	\item Apache Tika Core (MIME typ)
	\item JFoenix (material design)
	\item Gradle (build system)
\end{itemize}

%%%%%%%%%%%%%%%%%%%%%%%%%%%%%%%%%%%%%%%%%%%%%%%%%%%%%%%%%%%%%%%%%%%%%%%%%%%%%%
\section{Testovacie dáta}
Testovacie dáta budú prirodzene pozostávať z rôznych obrázkov rôzného formátu(JPG, PNG, BMP) a veľkosti. Obrázky budú mať rôzny počet kanálov(RGB, RGBA, GrayScale). Súčasťou testovacej sady budú napríklad aj súbory bez prípony, alebo súbory s nesprávnou príponou, aby bolo možné otestovať MIME typ obrázka nezávisle na prípone v názve súbora.
Testovacie dáta budú stiahnuté z nejakej voľne prístupnej databázy.

%%%%%%%%%%%%%%%%%%%%%%%%%%%%%%%%%%%%%%%%%%%%%%%%%%%%%%%%%%%%%%%%%%%%%%%%%%%%%%
\section{Aktuálny stav}
K dnešnému dátumu je hotové načítavanie obrázka a zobrazovanie týchto informácií:
\begin{itemize}
	\item informácie o súbore(názov, MIME typ, veľkosť v bytoch)
	\item informácie o obrázku(šírka a výška v pixeloch, bitová hĺbka, počet kanálov, celkový počet pixelov)
	\item informácie o pixele v mieste kurzora myši(x-ová a y-ová súradnica, farba v RGB)
	\item histogram rozdelený podľa RGB zložiek(jednotlivé zložky je možné schovávať a znova zobrazovať)
	\item !!! DOPLNIT !!!
\end{itemize}
\subsection{Screenshoty z aplikácie}

\section{Problémy, na ktoré sme stihli naraziť}
Funkcia \verb|imread()| s príznakom \verb|IMREAD_UNCHANGED| zhodí celý program, ak vstupom je 16-bitový \verb|tiff| súbor. Preto sme sa rozhodli vymeniť príznak \verb|IMREAD_UNCHANGED| za dvojicu príznakov \verb|CV_LOAD_IMAGE_ANYCOLOR| a \verb|CV_LOAD_IMAGE_ANYDEPTH|. Tým ale strácame informáciu o pôvodnom počte kanálov, pretože OpenCV pôvodný obrázok prekonvertuje do greyscale alebo RGB modelu a teda vždy pracujeme s jedným alebo troma 8-bitovými kanálmi. Pôvodná informácia o počte kanálov a bitovej hĺbke pre určité obrázky(napr. spomínaný 16-bitový \verb|tiff|) je teda nedostupná.

\newpage

%%%%%%%%%%%%%%%%%%%%%%%%%%%%%%%%%%%%%%%%%%%%%%%%%%%%%%%%%%%%%%%%%%%%%%%%%%%%%%
% Bibliografia
%%%%%%%%%%%%%%%%%%%%%%%%%%%%%%%%%%%%%%%%%%%%%%%%%%%%%%%%%%%%%%%%%%%%%%%%%%%%%%
%\begin{thebibliography}{9}
%	
%	\bibitem{ref1}
%	Wikipedia:
%	\emph{Otto Wiener (physics)} [Online; cit. 25-02-2017].
%	\newline
%	URL 
%	\url{https://en.wikipedia.org/wiki/Otto_Wiener}
%	
%	\bibitem{ref2}
%	Wikipedia:
%	\emph{Dagerotypia} [Online; cit. 25-02-2017].
%	URL 
%	\url{https://sk.wikipedia.org/wiki/Dagerotypia}
%	
%	\bibitem{ref3}
%	Skullsinthestars:
%	\emph{Classic Science Paper: Otto Wiener's experiment (1890)} [Online; cit. 25-02-2017]. URL \url{https://skullsinthestars.com/2008/05/04/classic-science-paper-otto-wieners-experiment-1890/}
%	
%	\bibitem{ref4}
%	Wikipedia:
%	\emph{Světlo} [Online; cit. 25-02-2017].
%	URL
%	\url{https://cs.wikipedia.org/wiki/Sv%C4%9Btlo}
%		
%		\bibitem{ref5}
%		e-Fyzika I:
%		\emph{Interferencia koherentných vlnení postupujúcich proti sebe, stojaté vlnenie} [Online; cit. 25-02-2017].
%		URL \url{http://kf-lin.elf.stuba.sk/~ballo/STU_online/Fyzika%20I/VI%20kapitola/kmity-vlny2-5-2.htm}
%			
%			\bibitem{ref6}
%			Encyklopedie fyziky:
%			\emph{Odraz vlnění v řadě bodů, stojaté vlnění} [Online; cit. 25-02-2017].
%			URL
%			\url{http://fyzika.jreichl.com/main.article/view/168-odraz-vlneni-v-rade-bodu-stojate-vlneni
%			}
%			
%			
%		\end{thebibliography}
\end{document}
