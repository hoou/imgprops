Ako implementačný jazyk bola zvolená java, ktorá so spojením gradle build systému poskytuje jednodúché používanie knižníc tretích strán. Jedna z najdostupnejších knižníc pre spracovanie obrazu je OpenCv. Pre javu existuje viacero implementácií.

\subsection{JavaCv}

Rozsiahla knižnica postavená na JavaCpp natívnych implementáciach.  Zabaľuje v sebe široké možnosti spracovania obrazu cez OpenCv a FFmpeg. Nepoužíva ale úplne rovnaké rozhranie pre OpenCv a jej dokumentácia nie je úplne kompletná \cite{bytedeco42:online}.

\subsection{OpenCv for Java}

Java knižnica využívajúca rovnaké rozhranie na OpenCv ako originálna implementácia v C++.  Jej hlavnou výhodou je  možnosť využitia oficiálnej dokumentácie k OpenCv \cite{openpnpo16:online}.