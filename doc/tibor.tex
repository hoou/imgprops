\subsection{Dostupné technológie}
Rozhodli sme sa navrhnutú aplikáciu implementovať v jazyku Java. Existuje niekoľko spôsobov ako v Jave pracovať s OpenCV knižnicou. Najvhodnejšími kandidátmi sú:
\begin{itemize}
	\item JavaCV
		\begin{itemize}
			\pro rozhranie pre OpenCV napísané v Jave
			\pro zahrňuje množstvo knižníc z oblasti počítačového videnia
			\con odlišné názvy funkcií od OpenCV
			\con horšia dokumentácia
		\end{itemize}
	\item OpenCV packaged by OpenPnP
		\begin{itemize}
			\pro práca s natívnou knižnicou OpenCV priamo v Jave (Java bindings)
			\pro chválitebná dokumentácia a množstvo príkladov
			\con \uv{len} OpenCV funkcie
		\end{itemize}
\end{itemize}

Pre získanie MIME typu súboru v Jave, bez ohľadu na príponu, je možné využiť niekoľko metód. Ako najspoľahlivejšie sa zdajú byť tieto:
\begin{itemize}
	\item jMimeMagic
	\begin{itemize}
		\pro spoľahlivé
		\pro jednoduché použitie
		\con otravné \uv{warning} oznámenia pri spustení
		\con nepodporuje TIFF formát
	\end{itemize}
	\item Apache Tika
	\begin{itemize}
		\pro mocný nástroj
		\pro dobrá podpora
		\pro vie to rozoznať tisíce rôznych formátov(aj TIFF!)
	\end{itemize}
\end{itemize}

\subsection{Informácie, ktoré je možné získať}
\begin{itemize}
	\item o súbore
	\begin{itemize}
		\item názov
		\item MIME typ
		\item veľkosť
	\end{itemize}
	\item o obrázku
	\begin{itemize}
		\item šírka a výška v pixeloch
		\item bitová hĺbka
		\item počet kanálov
		\item počet unikátnych farieb
	\end{itemize}
	\item o pixeli
	\begin{itemize}
		\item pozícia v obrázku
		\item farba, resp. intenzita
	\end{itemize}
	\item rôzne
	\begin{itemize}
		\item histogram
		\item rozdelenie obrázka na jednotlivé kanály
	\end{itemize}
\end{itemize}

\subsection{Podporované formáty obrázkov}
Navrhnutá aplikácia by mala podporovať tieto formáty obrázkov: BMP, JPG, PNG, TIFF. Dôležité zistenie je, že knižnica OpenCV nevie pracovať s GIF, tým pádom ani naša aplikácia to nebude podporovať.